\documentclass[a4paper]{article}
% \usepackage[margin=1.5in]{geometry}
\usepackage{fontspec}
\usepackage{csquotes}
\usepackage[hidelinks]{hyperref}
\usepackage[backend=biber, style=alphabetic]{biblatex}
\usepackage{lastpage}
\usepackage{xspace}
\usepackage{paralist}
\usepackage{caption}
\usepackage{subcaption}
\usepackage[english]{babel}
\usepackage{amsmath}
\usepackage{amssymb}
\usepackage{float}
\usepackage{mathtools}
\usepackage{lipsum}
\usepackage[svgnames]{xcolor}
\usepackage{caption}
\usepackage{subcaption}
\usepackage{enumitem}

\makeatletter
\renewcommand{\@evenfoot}{\hfil \thepage{} of \pageref*{LastPage}\hfil}
\renewcommand{\@oddfoot}{\@evenfoot}
\makeatother

\newcommand{\tname}[1]{\textsc{#1}\xspace}
\newcommand{\etcd}{\tname{etcd}}

\newcommand*{\eg}{e.g.\@\xspace}
\newcommand*{\ie}{i.e.\@\xspace}

\def\sectionautorefname{Section}
\def\subsectionautorefname{Section}

\bibliography{references}

\title{CS6213 Project Report}
\author{George Pîrlea \and Darius Foo}
\date{\today}

\begin{document}

\maketitle

\section{Motivation}

Writing a formal specification of a system is useful in clarifying details of its design.
%
As the system evolves, however, it is difficult to ensure that its implementation and specification have not diverged.

In this work, we investigate lightweight (an engineer can learn and produce something useful in 2-3 weeks~\cite{newcombe_how_2015}) but systematic methods to check that a system implementation conforms to its specification.

\section{Scope}

We set out to answer the following research questions:

\begin{enumerate}[label={(Q\arabic*)}]
    \item \textbf{Conformance testing.} Given a TLA model that satisfies some simple properties and a system implementation, what is the easiest way to check that the implementation satisfies the properties?

    \item \textbf{Refinement mappings.} Given a model and an implementation, how can we best define the correspondence between the two, perhaps via a combination of manual/inferred annotations?

    \item \textbf{Empirical evaluation of conformance.} For real-world distributed systems that come with both TLA models and implementations, empirically, is there a correspondence between the model and the implementation?
\end{enumerate}

Our original intention was to select three open-source implementations of the Raft consensus protocol~\cite{ongaro_search_2014, ongaro_consensus_2014} and evaluate to what extent they match the behaviour of the TLA specification written by Diego Ongaro.

Due to time constraints, we had to reduce our ambitions.
%
In the end, we evaluated only the most widely used implementation, \etcd, and not to the extent we had initially hoped.
%
Nonetheless, our experience has been instructive and leads us to believe there is potential in continuing this line of work.

\section{Project Summary}

\begin{itemize}
    \item We combined three existing TLA specifications (due to Diego Ongaro, Daniel Ricketts, and Brandon Amos and Hangcheng Zhang) into a reference specification of Raft with one-at-a-time membership changes.

    \item We started developing a prototype testing framework on top of \etcd to simulate execution traces generated by TLC.

    \item We noticed that model checking performance is low.
    %
    With the default TLC configuration, on a model with 3 nodes, it took 2 minutes before the first entry was added to a log, when running TLC on our laptops.
    %
    Entries began to be committed only much later.

    \item Hoping for a better model checking experience, we adapted our specification for use with the Apalache symbolic model checker.
    %
    This required adding type annotations to the TLA specification and slightly rewriting some operators to fit into the subset of TLA that can be automatically translated to SMT.
    %
    Apalache is alpha software, so we ran into a large number of bugs.
    %
    Sadly, for our specification, Apalache symbolic model checking was orders of magnitude slower than TLC.

    \item Disappointed with Apalache's performance, we reverted to using TLC to generate Raft execution traces.
    
    \item We investigated approaches to constrain TLC's search space (state constraints, symmetry, action constraints, and views).
    %
    We also explored different search strategies (depth first search, simulation mode), but these were unhelpful.
    %
    Put together, the various approaches offered a 2-3 orders of magnitude performance improvement for our use-case of generating test traces.
    % 
    But even this was not enough to generate traces for all the scenarios we wanted to test, some of which required traces 30 steps long.

    \item We discovered that, for trace generation (as opposed to model checking), certain properties allow for a specialised search strategy reminiscent of concolic execution: a burst of breadth-first search, followed by focusing on a single trace (and its final state), followed by another breadth-first search starting from \emph{that} state.
    %
    We used this strategy, which we ``implemented'' manually using TLC state constraints, to obtain very deep Raft traces, \eg, a period with two concurrent leaders during which an entry commits, followed by a majority of the cluster restarting.

    \item In the process, we discovered (low impact) errors in the
    specification of membership changes due to Amos and Zhang and in the
    properties stated by Daniel Rickkets.
    %
    This suggests that their specifications have been model-checked only to a shallow extent.
    %
    Given the effort we spent to obtain reasonable performance out of TLC, this is understandable.

\end{itemize}


\section{Approach}

\subsection{Model-based Test Case Generation (MBTCG)}
\label{sec: interpreter}

The MBTCG approach uses a specification to guide the generation of test cases for the implementation.

The specification is typically given in the form of a transition system (consisting of states and \emph{actions} which give rise to transitions between states) and its invariants.
%
A model checker may then be used to output sequences of states which violate the invariants.
%
Such sequences may equivalently be seen as sequences of actions, starting and ending at particular states.

The idea is then to feed a sequence of actions into the implementation and check if the implementation also arrives at the final state.
%
To do that, the implementation must be extended with

\begin{enumerate}
    \item An abstraction function, mapping an implementation state to a specification state.

    \item An interpreter for actions. In some sense this may be seen as a concretization function, mapping a specification action to an implementation state change.
\end{enumerate}

There are a number of challenges to doing this.

\begin{itemize}
\item \textbf{Concurrency.} The interpreter has to drive the implementation forward in such a way that each intermediate state corresponds to a linearization point.
%
There is otherwise no guarantee that any observed implementation state matches some specification state.
%
To ensure this, the interpreter is synchronised, e.g. waiting for a message to appear in the soup as confirmation that an action has been executed.
%
At that point we know that the system is relatively quiescent and can observe its state.

\item \textbf{Modifications.} Implementing the interpreter might require changes to the implementation which interfere with its integrity: to what extent is the modified implementation representative of the program we want to check the conformance of?
%
This is of particular concern when the interpreter is retroactively added to an implementation, as we are doing; ideally, the implementation would natively support interpreting actions.
%
We try to minimize our changes and limit them to events that may be externally controlled, e.g. timeouts and message sends and receives, and implement only what is necessary to expose external control of such events.

% this whole section could probably be explained better, with an example

\item \textbf{Nondeterminism.} Nondeterministic specification actions typically correspond to some composition of implementation actions with a deterministic prefix.
%
For example, a specification action might require that a message is sent when a variable has a certain value; an implementation is likely to send this message upon the variable's value changing as part of an earlier action.
%
Decoupling these actions in the implementation is not always possible (if determinism is required) or desirable (if it would be a nontrivial implementation change).
%
Not decoupling them would risk the linearization point being misplaced, but we choose this option instead, with a number of mitigating measures:

\begin{enumerate}
\item The specification could be rewritten to remove nondeterminism, so that actions have a deterministic prefix.
%
This is always possible because sequences of actions generated by the model checker can be reordered if the actions commute, and if the specification actions corresponding to the composed implementation transition are the only ones modifying some particular part of the state, they would commute with all intervening ones and could be reordered and fused \cite{lipton1975reduction}.

\item The abstraction function could be used to remove state which would cause the linearization point to be misplaced.
\end{enumerate}

\end{itemize}

% manual work required to create a test case:

% - manually removed messages nodes send to themselves (there's no analogue in the impl)
% - extra action manually added at the end, to ensure that we've waited long enough for the state to quiesce before inspecting it
% - the subparts of the spec and impl state that we want to compare need to be defined for each kind of test

\subsubsection{Takeaways}

It is not enough that the specification describes the essence of the implementation; it must be \emph{expressed} in such a way that the correspondence with the implementation is straightforward.
%
This conclusion agrees with prior work \cite{Davis_2020}.
%
Dually, the implementation should also support the execution of high-level actions for this technique to be easily applied.

% \renewcommand*{\bibfont}{\footnotesize}
\printbibliography

\end{document}
